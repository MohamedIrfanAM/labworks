\experiment{Polynomial Linked List}{Your_Date_Here}

\section{Aim}
To implement a program that performs operations on polynomials, including multiplication and addition.

\section{Algorithm}
 {\fontfamily{lmtt}\selectfont

  \subsection{Structure Definition}
  Create a structure \texttt{Node} with the following attributes:
  \begin{enumerate}[label=\arabic*:,left=0pt]
    \item Integer \texttt{exp} to store the exponent of the term.
    \item Integer \texttt{coeff} to store the coefficient of the term.
    \item Pointer to \texttt{Node} \texttt{next} for linking nodes.
  \end{enumerate}

  \subsection{Function: \texttt{insertBegin}}
  Create a function \texttt{insertBegin(head, coeff, exp)}:
  \begin{enumerate}[label=\arabic*:,left=0pt]
    \item \textbf{Start}
    \item Allocate memory for a new \texttt{Node} structure using \texttt{malloc}.
    \item If allocation fails, print "Memory allocation failed" and exit.
    \item Set \texttt{coeff} and \texttt{exp} in the new node.
    \item Set \texttt{next} of the new node to \texttt{head}.
    \item Set \texttt{head} to the new node.
    \item Return the updated \texttt{head}.
    \item \textbf{Stop}
  \end{enumerate}

  \subsection{Function: \texttt{bubblesort}}
  Create a function \texttt{bubblesort(head, n)}:
  \begin{enumerate}[label=\arabic*:,left=0pt]
    \item \textbf{Start}
    \item Loop from \texttt{i} equal to \texttt{n - 1} down to 0:
          \begin{enumerate}[label=2.\arabic*:, start=1]
            \item Set \texttt{current} to \texttt{head}.
            \item Loop from \texttt{j} equal to 0 to \texttt{i}:
                  \begin{enumerate}[label=2.2.\arabic*:, start=1]
                    \item Set \texttt{next} to \texttt{current->next}.
                    \item If \texttt{current->exp > next->exp}, swap the \texttt{coeff} and \texttt{exp} values.
                    \item Set \texttt{current} to \texttt{next}.
                  \end{enumerate}
          \end{enumerate}
    \item \textbf{Stop}
  \end{enumerate}

  \subsection{Function: \texttt{display}}
  Create a function \texttt{display(head)}:
  \begin{enumerate}[label=\arabic*:,left=0pt]
    \item \textbf{Start}
    \item Set \texttt{current} to \texttt{head}.
    \item Loop while \texttt{current} is not \texttt{NULL}:
          \begin{enumerate}[label=2.\arabic*:, start=1]
            \item If \texttt{current} is not equal to \texttt{head}, print " + ".
            \item Print \texttt{current->coeff} and \texttt{current->exp}.
            \item Set \texttt{current} to \texttt{current->next}.
          \end{enumerate}
    \item Print a newline.
    \item \textbf{Stop}
  \end{enumerate}

  \subsection{Function: \texttt{multiply}}
  Create a function \texttt{multiply(a, b)}:
  \begin{enumerate}[label=\arabic*:,left=0pt]
    \item \textbf{Start}
    \item Set \texttt{result} to \texttt{NULL}.
    \item Set \texttt{currenta} to \texttt{a}.
    \item Set \texttt{n} to 0.
    \item Loop while \texttt{currenta} is not \texttt{NULL}:
          \begin{enumerate}[label=2.\arabic*:, start=1]
            \item Set \texttt{currentb} to \texttt{b}.
            \item Loop while \texttt{currentb} is not \texttt{NULL}:
                  \begin{enumerate}[label=2.2.\arabic*:, start=1]
                    \item Calculate \texttt{coeff} and \texttt{exp} by multiplying \texttt{currenta->coeff} with \texttt{currentb->coeff} and adding \texttt{currenta->exp} with \texttt{currentb->exp}.
                    \item Set \texttt{currentr} to \texttt{result}.
                    \item Set \texttt{alreadyExist} to 0.
                    \item Loop while \texttt{currentr} is not \texttt{NULL}:
                          \begin{enumerate}[label=2.2.3.\arabic*:, start=1]
                            \item If \texttt{currentr->exp == exp}, update \texttt{currentr->coeff} by adding \texttt{coeff} and set \texttt{alreadyExist} to 1.
                            \item Set \texttt{currentr} to \texttt{currentr->next}.
                          \end{enumerate}
                    \item If \texttt{alreadyExist} is 0, increment \texttt{n}, and insert a new node with \texttt{coeff} and \texttt{exp} at the beginning of \texttt{result}.
                  \end{enumerate}
            \item Set \texttt{currentb} to \texttt{currentb->next}.
          \end{enumerate}
    \item Call \texttt{bubblesort(result, n)}.
    \item \textbf{Stop}
  \end{enumerate}

  \subsection{Function: \texttt{add}}
  Create a function \texttt{add(a, b)}:
  \begin{enumerate}[label=\arabic*:,left=0pt]
    \item \textbf{Start}
    \item Set \texttt{result} to \texttt{NULL}.
    \item Set \texttt{currenta} to \texttt{a}.
    \item Set \texttt{currentb} to \texttt{b}.
    \item Set \texttt{n} to 0.
    \item Loop while \texttt{currenta} is not \texttt{NULL} or \texttt{currentb} is not \texttt{NULL}:
          \begin{enumerate}[label=2.\arabic*:, start=1]
            \item If \texttt{currenta} and \texttt{currentb} are not \texttt{NULL}:
                  \begin{enumerate}[label=2.2.\arabic*:, start=1]
                    \item If \texttt{currenta->exp == currentb->exp}, calculate \texttt{coeff} by adding \texttt{currenta->coeff} and \texttt{currentb->coeff} and insert a new node with \texttt{coeff} and \texttt{currenta->exp} at the beginning of \texttt{result}.
                    \item Move \texttt{currenta} and \texttt{currentb} to their next nodes.
                  \end{enumerate}
            \item If only \texttt{currenta} is not \texttt{NULL}, insert a new node with \texttt{currenta->coeff} and \texttt{currenta->exp} at the beginning of \texttt{result}.
            \item If only \texttt{currentb} is not \texttt{NULL}, insert a new node with \texttt{currentb->coeff} and \texttt{currentb->exp} at the beginning of \texttt{result}.
            \item Increment \texttt{n}.
          \end{enumerate}
    \item Call \texttt{bubblesort(result, n)}.
    \item \textbf{Stop}
  \end{enumerate}

  \subsection{Main Function}
  In the \texttt{main} function:
  \begin{enumerate}[label=\arabic*:, start=1]
    \item \textbf{Start}
    \item Declare \texttt{a} and \texttt{b} as \texttt{NULL}.
    \item Declare integers \texttt{n1} and \texttt{n2}.
    \item Print "Enter the number of terms of a: ".
    \item Take user input for \texttt{n1} and insert nodes at the beginning of \texttt{a} with user-input coefficients and exponents.
    \item Print "Enter the number of terms of b: ".
    \item Take user input for \texttt{n2} and insert nodes at the beginning of \texttt{b} with user-input coefficients and exponents.
    \item Call \texttt{bubblesort(a, n1)} and \texttt{bubblesort(b, n2)}.
    \item Print "a = " and \texttt{display(a)}.
    \item Print "b = " and \texttt{display(b)}.
    \item Call \texttt{multiply(a, b)} and assign the result to \texttt{product}.
    \item Print "Product = " and \texttt{display(product)}.
    \item Call \texttt{add(a, b)} and assign the result to \texttt{sum}.
    \item Print "Sum = " and \texttt{display(sum)}.
    \item \textbf{Stop}
  \end{enumerate}
  \textbf{End Algorithm}
 }

\section{C Program}
\begin{lstlisting}[label={list:c_program:polynomial_operations}]
#include <stdlib.h>
#include <stdio.h>

typedef struct Node {
    int exp;
    int coeff;
    struct Node *next;
} node;

// Function declarations...

int main() {
    // Implementation...
}
\end{lstlisting}
