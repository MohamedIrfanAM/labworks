\experiment{Polynomial Addition}{10/09/2023}


\section*{Algorithm}

\begin{algorithm}
    \item Read the degree of the polynomial and store it in a variable n.sdkfljasldkfja;sdlfkjas;dlfkaj;dfl\\
    (a) kasjdfl;ksjdf;lkajsd;flakjsd;kjd\\
    fa;ldkfjas;ldkfjs;ldkfj;
    \item Create an array poly of size n+1 to store the coefficients of the polynomial.
    \item Read the coefficients of the polynomial from the user and store them in the array poly.
    \item To evaluate the polynomial at a given value of x, use the following formula:
\end{algorithm}


\section*{C Program}

\begin{lstlisting}[label={list:first}]
#include <stdio.h>

int main() {
    int num1, num2, sum;
    printf("Enter two numbers: ");
    scanf("%d %d", &num1, &num2);
    sum = num1 + num2;
    printf("The sum of %d and %d is %d.", num1, num2, sum);
    return 0;
}
\end{lstlisting}


\section*{Output}

\begin{lstlisting}[label={list:output}]
Enter two numbers: 5 7
The sum of 5 and 7 is 12.
\end{lstlisting}


\section*{Result}
Program is executed successfully and output is verified